\chapter{Teaching \cm Cosmology to the Public}\label{Ch:Planet}
Cosmology using the \cm line of hydrogen is a relatively new concept in the world of astronomy. It relies on radio telescopes, which have only been in general use for $\sim 50$ years. Measurements of the \cm signal from our own Milky Way Galaxy and other nearby galaxies have been used to study the neutral hydrogen gas distribution around the galaxies. 

On cosmological scales, where the \cm signals are $10^4 - 10^6$ times smaller than the foregrounds, the field is still in its infancy. There are many telescopes seeking to measure the \cm signal at redshifts ($z>1$), but as of yet, there has not been a significant detection in any of these cosmological redshift bands. 


\section{Overview}
One of the National Science Foundation grants that supports the Peterson group (\textcolor{red}{Add NSF reference number here.}) included \$20,000 for the creation of a $5-10$ minute planetarium show entitled $''$The Hydrogen Sky$''$. As producer, I have been working with a team of animators and artists to develop this show. I recruited two animators from Carnegie Mellon University's Entertainment Technology Center masters program\footnote{http://www.etc.cmu.edu/}, as well as working with a local Pittsburgh animation company, Home Run Pictures\footnote{http://www.hrpictures.com/}. 



\section{Storyboard and Script}

\subsection{Introduction}
\subsection{\cm Science}
\subsection{Radio Telescopes}
\subsection{Green Bank Telescope and Intensity Mapping}
\subsection{CHIME Telescope}
\subsection{Science Goals}

\section{On-site Filming}

\subsection{Green Bank, WV}
\subsection{Penticton, BC, Canada}

\section{Working in the Planetarium Environment}

\subsection{Technology}
\subsection{Visual Differences}

\section{Audio Production}

\subsection{Narration}
\subsection{Music and Sound Effects}

\section{Feedback and Evaluation}

