\chapter{Looking Toward the Future}\label{Ch:Conclude}

\section{Deployment of the Improved SCI-HI System}

Current limits with the SCI-HI system from the data collected in June 2013, as shown in Chapter \ref{Ch:Data}, are above the threshold needed to measure the \cm signal during the Cosmic Dawn. In addition, the results only cover a relatively narrow frequency band. Limits are currently dominated by contamination from self-generated RF and RFI in the FM band. 

Self-generated RF comes from instrumental defects including insufficient Faraday Cage shielding, limited antenna bandwidth, and power sources (batteries) that last less than 24 hours. These issues can can be addressed by building improvements to the SCI-HI system. Improvements include the new HIbiscus antennas discussed in Section \ref{Sec:HIant_upgrade} to expand the antenna bandwidth. Also, additional Faraday Cage shielding, discussed in Section \ref{Sec:sys_noise}, and a new power source have been added to the system. This new power source is a gasoline fueled generator, which is more reliable than batteries, as discussed in Section \ref{Sec:sys_power}. In addition, the data processing system computer has been upgraded since the June 2013 deployment. The new system has a duty cycle of 30\%, compared to the previous duty cycle of 5-10\%.  

We plan to deploy this updated system to the SKA site in the Karoo desert in April 2015, as discussed in Section \ref{Sec:SA_site}. During the Karoo deployment, data will be collected for full 24 hour cycles using each of the two HIbiscus antennas. This will allow us to optimize the Milky Way Galaxy calibration strategy and utilize a larger range of frequencies. Given the improvements to the SCI-HI system, we expect that self-generated RF contributions to the sky temperature will be decreased to levels below the \cm signal. In addition, the higher duty cycle will allow us to collect enough data to bring $P_{thermal}$ below the \cm signal levels with only a few days of data. 

However, the Karoo desert site is not expected to be sufficiently remote to allow use of the FM band. Therefore, we plan to deploy the SCI-HI system to Marion Island in April 2016, as discussed in Section \ref{Sec:SA_site}. The RFI levels in the FM band at this site are expected to be below the current residual limits measured by the SCI-HI system. 

Deployment at the Karoo desert should yield data with self-generated RF for $f<88 MHz$ to below \cm signal levels. Once this component in the data is removed, the residuals will be dominated by one of three components. These three components are: (a) ionospheric refraction and absorption, (b) variance in the spectral index of the Milky Way Galaxy, and (c) the \cm signal. 

Data collected in the Karoo will allow us to identify which contribution (a, b or c) is dominant in the residuals between ($40\leq f \leq 88 MHz$). If the \cm signal is the dominant residual in the Karoo data at those frequencies, we will make a first detection of the \cm signal. Otherwise, we will be able to place tighter constraints on the \cm signal. The level of these constraints will be set by the level of the dominant component in the residuals. 

\section{Development of Expanded Global \cm Signal Experiments}

Funding for the Karoo and Marion deployments is part of a three-year grant from the South African National Antarctic Program (SANAP). The grant is also intended to support further development of the SCI-HI experiment, which will be called $''$SCI-HI in the Sub-Antarctic$''$ (SHISA). 

SHISA plans to continue improving the SCI-HI system design. Some potential areas of improvement include a multi-element version of the system for improved spatial resolution, full polarization data collection, and a more sophisticated data processing system. These developments will allow us to address contributions to the residuals from the ionosphere and foregrounds at levels beyond the capability of the current SCI-HI system.

One of the advantages to the Marion Island site is the potential for diminished ionospheric impacts, particularly during the Antarctic winter. As part of the SHISA project, While we are on the island in April 2016, we plan to deploy a system for evaluating the ionosphere at Marion Island's latitude during the entire year. This system will focus on frequencies $\sim5-50 MHz$, where the ionospheric impact is large. Using the frequency spectrum of the signal we will be able to identify the lowest frequency where we can still see the sky through the ionosphere as a function of time. This will allow us to better quantify the ionospheric impacts on global \cm experiments and other low frequency radio astronomy projects deployed to the island. 