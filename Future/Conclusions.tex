\chapter{Looking toward the Future}\label{Ch:Conclude}

Current limits with the SCI-HI system from the data collected in June 2013, as shown in Chapter \ref{Ch:Data}, are above the threshold needed to measure the \cm signal during the Cosmic Dawn. In addition, the results only cover a relatively narrow frequency band. This is due to contamination in the data and the limitations of the system. 

We have been building improvements to the SCI-HI system based on the problems identified through the June 2013 deployment. These improvements include the new HIbiscus antennas discussed in Section \ref{Sec:HIant_upgrade}. In addition, the data processing system computer has been upgraded such that the system duty cycle is now 30\% compared to $\sim$10\% for data collected in June 2013. The upgraded system has a new Faraday cage with additional shielding as discussed in Section \ref{Sec:sys_noise}. 

As mentioned in Chapter \ref{Ch:RFI}, the improved SCI-HI system will be deployed on Marion Island in April 2015. Data will be collected for full 24 hour cycles using each of the two antennas in order to optimize the Milky Way Galaxy calibration strategy and utilize the entire frequency band. Following deployment, the data will be analyzed. Assuming that there are no surprises, the updated system should allow us to reach residuals below the $\sim100$ mK threshold needed for a first detection (or constraints) of the \cm signal over the entire $40 \leq \nu \leq 130$ MHz band. 

Funding for the Marion deployment is part of a three-year grant from the South African National Antarctic Program (SANAP). The grant is also intended to support further development of the SCI-HI experiment, which will be called $''$SCI-HI in the Sub-Antarctic$''$ (SHISA). 

SHISA plans to continue improving the SCI-HI system design. Some potential areas of improvement include a multi-element version of the system for improved spatial resolution, full polarization data collection, and a more sophisticated data processing system. 

One of the advantages to the Marion Island site is the potential for lowered ionospheric impacts during the Antarctic winter. We plan to deploy a system for evaluating Marion Island during the winter while on the island in April. This system will focus on frequencies down to $\sim$5 MHz, allowing us to assess the Earth's ionosphere and identify the change in the minimum useable frequency over the winter. 
