\chapter{Introduction}\label{Ch:Intro}

\section{Basics of \cm Observations}
Foundational to any discussion of \cm cosmology is the basic mechanism of absorption and emission of the \cm hyperfine line of hydrogen atoms. 

\textcolor{red}{How far back to the fundamentals do I need to go for this context? In other words, what starting point is assumed for a thesis?}

\textcolor{blue}{First few paragraphs would set the atomic physics of hydrogen. Depending on the necessary depth, I could go into a discussion of the Hydrogen atom.}

\textcolor{blue}{Next few paragraphs would be a description of the process of stimulated emission or absorption that occurs when \cm photons pass through a cloud of hydrogen atoms.}

\textcolor{red}{Much of this discussion can be gleaned from the field paper and the furlanetto oh and briggs review.}

\section{History of the Intergalactic Medium}
Once we understand the mechanism of \cm emission and absorption, we can start to consider the history of the intergalactic medium (IGM) and the hydrogen contained in that medium. In particular, we will focus on a few key eras during which the state of the hydrogen changed. 

\textcolor{red}{Need to have a discussion of how the IGM can be traced using the \cm spin temperature and the dependencies of the spin temperature. }

The spin temperature is coupled to three sources during its history. The first source is the CMB (\tg \textcolor{red}{Have I defined this yet?}), which provides the backlight for the IGM. The second source is the kinetic temperature of the gas (\tk), which characterizes the thermal motion of the atoms in the gas. The third and final source is external photons corresponding to the transitions between states of the hydrogen atom. 

The Lyman-$\alpha$ transition between the ground state of the atom and its first excited state is the primary transition that impacts the spin temperature. So, we can describe the external photon source as a Lyman-$\alpha$ temperature (\tl). Given these three sources, the spin temperature can be defined by Equation \ref{Eq:T_s} where $x_k$ and $x_{Ly-\alpha}$ are the coupling terms. 

\begin{equation}\label{Eq:T_s}
T_s = \frac{T_{\gamma} + x_k T_{k} + x_{Ly-\alpha} T_{Ly-\alpha}}{1+x_k +x_{Ly-\alpha}}
\end{equation}

\subsection{Epoch of Recombination}
\textcolor{red}{Not sure how much detail I should put in about the epoch of recombination, since it's not directly applicable to my work.}

\textcolor{blue}{Discussion here should include how the intergalactic medium (aka everything at this point) is dominated by neutral hydrogen atoms. }

\textcolor{blue}{I also need to discuss the Cosmic Microwave Background radiation as a backlight from this era.}

\subsection{Dark Ages}
During the dark ages, the universe continued to expand. This expansion caused the kinetic temperature of the gas to decrease below the CMB temperature. Initially, the coupling between \ts and \tk meant that the spin temperature also decreased (\ts$<$\tg). However, this coupling gradually decreased (\textcolor{red}{Why did this happen?} and the spin temperature increased back to match the CMB temperature. The process of cooling and decoupling created a dip in \ts, centered around $z=$\textcolor{red}{redshift?}. 

At the same time, the first generation of stars began to form in the overdense regions of space. 
\subsection{Cosmic Dawn}


\subsection{Epoch of Reionization}

\subsection{Era of Acceleration}

\section{The \cm Global Spectrum}

\section{\cm Spatial Structure and Intensity Mapping}

