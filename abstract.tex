\begin{abstract}

The Cosmic Dawn ($z\sim~15-35$) is the period in the history of our universe when the stars first began to form in the centers of galaxies. Light from these first stars is too dim for telescopes to see, which means that the Cosmic Dawn has never been directly studied. However, the first stars affected the gas (IGM) around them, heating and eventually ionizing the IGM. This process of heating and ionization creates a spatially averaged brightness temperature spectrum of \cm light from the IGM that varies over redshift. Measurement of the spatially averaged \cm spectrum will give us a first glimpse of the Cosmic Dawn. 

The $''$Sonda Cosmologica de las Islas para la Deteccion de Hidrogeno Neutro$''$ (SCI-HI) experiment is a collaboration between Carnegie Mellon University (CMU) and Instituto Nacional de Astrof\'{i}sica, \'{O}ptica y Electr\'{o}nica (INAOE) in Mexico and was designed to make this measurement. The SCI-HI experiment is a small-scale system which travels to remote locations for deployments. These remote locations are necessary to avoid radio frequency interference and other environmental impacts on the system.

This thesis describes the development and deployment of the SCI-HI experiment. It starts with the original design and covers development of the system over time. Deployment location selection is discussed, including the results of site evaluations. In addition, the thesis outlines the data analysis process used for the system and shows results from data collected during the June 2013 deployment of the experiment. Finally, the thesis describes the plans for the future of the SCI-HI experiment, including deployment to South Africa in 2015. 

\end{abstract}