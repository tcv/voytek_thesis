\begin{abstract}

The Cosmic Dawn ($z\sim~15-35$) is the period in our history when the stars first began to form in the centers of the first galaxies in the universe. Light from these first stars is too dim for telescopes to see, which means that the cosmic dawn has never been directly studied. However, the first stars affected the gas (IGM) around them, heating and eventually ionizing the IGM. This process of heating and ionization creates a brightness temperature spectrum of \cm light from the IGM that varies over redshift. Measurement of the \cm spectrum will give us a first glimpse of the cosmic dawn. 

The $''$Sonda Cosmologica de las Islas para la Deteccion de Hidrogeno Neutro$''$ (SCI-HI) experiment is a collaboration between Carnegie Mellon University (CMU) and Instituto Nacional de Astrof\'{i}sica, \'{O}ptica y Electr\'{o}nica (INAOE) in Mexico and was designed to make this measurement. This thesis describes the SCI-HI experiment, including system design, deployment, data analysis, first results, and plans for the future. 

\end{abstract}